\documentclass[12pt]{article}

\usepackage[sfdefault]{ClearSans}
\usepackage[utf8]{inputenc}
\usepackage[T1]{fontenc}
\usepackage[francais]{babel}
\usepackage{color}
\usepackage[top=2cm, bottom=2cm, left=2cm, right=2cm]{geometry}
\usepackage{eurosym}
\usepackage{graphicx}
\usepackage{fancybox}
\usepackage{listings}

\definecolor{codegreen}{rgb}{0,0.6,0}
\definecolor{codegray}{rgb}{0.5,0.5,0.5}
\definecolor{codepurple}{rgb}{0.58,0,0.82}
\definecolor{backcolour}{rgb}{0.95,0.95,0.92}

\lstdefinestyle{mystyle}{
    backgroundcolor=\color{backcolour},
    commentstyle=\color{codegreen},
    keywordstyle=\color{magenta},
    numberstyle=\color{codegray},
    stringstyle=\color{codepurple},
    breakatwhitespace=false,
    breaklines=true,
    captionpos=b,
    keepspaces=true,
    numbers=left,
    numbersep=3pt,
    showspaces=false,
    showstringspaces=false,
    showtabs=false,
    tabsize=2
}

\lstset{style=mystyle}

\pagestyle{plain}
\title{Sémaphore local}
\author{Anthony Araye et Camille Schnell}
\date{21 février 2018}
\begin{document}
\maketitle
\renewcommand{\contentsname}{Sommaire}
\tableofcontents
\newpage
\section{Introduction}
    En informatique, lorsque deux programmes exécutés en parallèle souhaitent accéder à la même donnée au même moment, cela peut créer des conflits, notamment si l'un des deux modifie la donnée. \\

    \textit{\textbf{LocalSemaphores}} permet l'implémentation d'un système de sémaphores inter processus hérités par les fils d'un processus afin de gérer des sections critiques.

    Les sémaphores peuvent être initialisés avec un nombre de ressources donné (fonction \textit{sem\_initialize(int nb)}), et gèrent ensuite l'allocation des ressources et la file d'attente des processus bloqués grâce aux fonctions \textit{sem\_acquire(int sid)} et \textit{sem\_release(int sid)}. La fonction \textit{sem\_destroy(int sid)} permet de détruire un sémaphore et de désallouer la file d'attente associée à ce dernier.

    Un sémaphore est détruit lorsque le processus qui l'a créé ainsi que tous ses fils meurent. Sa non-rémanence lui permet ainsi d'être plus efficace que les IPC (Inter Processus Communication) utilisés habituellement sous linux.

\newpage
\section{Manuel d'utilisation}
    \subsection{Appels systèmes}
      \subsubsection{initialize}
          \paragraph{Nom\\}
          \textbf{\#include <sem.h>}

          \textbf{int sem\_initialize(int }\textit{nb}\textbf{)}
          \paragraph{Description\\}
          \textbf{sem\_initialize}() alloue et initialise un sémaphore avec \textit{nb} comme valeur initiale. \\ \textit{nb} représente le nombre de ressources disponibles. Cette valeur doit être supérieure ou égale à 0, elle vaut 1 par défaut.

          \paragraph{Valeur renvoyée\\}
          \textbf{sem\_initialize}() renvoie l'id du sémaphore dans le cas où celui-ci a bien été créé, -1 si une erreur a été levée.
          \paragraph{Erreurs\\}
          ENOMEM : lorsque l'allocation ne s'est pas faite. \\
          EFAULT : lorsque la valeur passée en argument est invalide (\textit{nb} < 0 ou \textit{nb} > MAX\_INT). \\
          ETOOMANY : lorsqu'on a atteint le nombre limite de sémaphores par processus. \\
      \newpage
      \subsubsection{acquire}
          \paragraph{Nom\\}
          \textbf{\#include <sem.h>}

          \textbf{int sem\_acquire(int }\textit{sid}\textbf{)}
          \paragraph{Description\\}
          \textbf{sem\_acquire} décrémente la valeur du sémaphore d'id \textit{sid} si le nombre de ressources disponibles est non nul. Dans le cas où cette valeur est inférieur ou égale à 0, alors il bloque le processus courant et le met dans la file d'attente associée au sémaphore.

          \paragraph{Valeur renvoyée\\}
          \textbf{sem\_acquire} renvoie 0 dans le cas où le processus s'est bien effectué, -1 si une erreur a été levée.
          \paragraph{Erreurs\\}
          ENOMEM : lorsque l'allocation ne s'est pas faite.

          EFAULT : lorsque le \textit{sid} passé en argument ne correspond pas à un sémaphore existant.
      \newpage
      \subsubsection{release}
          \paragraph{Nom\\}
          \textbf{\#include <sem.h>}

          \textbf{int sem\_release(int }\textit{sid}\textbf{)}
          \paragraph{Description\\}
          \textbf{sem\_release} incrémente le nombre de ressources disponibles du sémaphore d'id \textit{sid}. Dans le cas où la file associée à \textit{sid} est non vide, alors il réveille le processus en tête de file et saute la tête de la file.

          \textbf{sem\_release} ne peut être appelé que quand au moins une ressource est utilisée.

          \paragraph{Valeur renvoyée\\}
          \textbf{sem\_release} renvoie 0 dans le cas où le processus s'est bien effectué, -1 si une erreur a été levée.
          \paragraph{Erreurs\\}
          ENOMEM : lorsque l'allocation ne s'est pas faite.

          EFAULT : lorsque le \textit{sid} passé en argument ne correspond pas à un sémaphore existant.
      \newpage
      \subsubsection{destroy}
          \paragraph{Nom\\}
          \textbf{\#include <sem.h>}

          \textbf{int sem\_destroy(int} \textit{sid}\textbf{)}
          \paragraph{Description\\}
          \textbf{sem\_destroy} détruit le sémaphore d'id \textit{sid} et désalloue la file d'attente.

          \paragraph{Valeur renvoyée\\}
          \textbf{sem\_destroy} renvoie 0 dans le cas où le sémaphore a bien été détruit, -1 si une erreur a été levée.
          \paragraph{Erreurs\\}
          EFAULT : lorsque le \textit{sid} passé en argument ne correspond pas à un sémaphore existant.
        \newpage
        \subsubsection{debug}
            \paragraph{Nom\\}
            \textbf{\#include <sem.h>}

            \textbf{int sem\_debug(int} \textit{sid}\textbf{, int* }\textit{v}\textbf{, int* }\textit{pids}\textbf{, [int }\textit{pidSize}\textbf{], int }\textit{compteurDeReference}\textbf{)}
            \paragraph{Description\\}
            \textbf{sem\_debug} permet de connaître le nombre de processus dans la file d'attente du sémaphore d'id \textit{sid}, utile lors d'un debug.

            \textit{v} représente le nombre de ressources, \textit{pids} représente les processus dans la file d'attente, \textit{pidSize} représente la taille du tableau \textit{pids}.

            \paragraph{Valeur renvoyée\\}
            \textbf{sem\_debug} renvoie le nombre de processus dans la file d'attente du sémaphore d'id \textit{sid}, -1 si une erreur a été levée.
            \paragraph{Erreurs\\}
            EFAULT : lorsque le \textit{sid} passé en argument ne correspond pas à un sémaphore existant.
    \newpage
    \subsection{Exemple}
    \begin{lstlisting}[language=C]
    int main(int argc, char* argv) {
      int s1, s2, p1, p2, status, pid1, pid2;

      // init semaphores
      if(s1 = sem_initialize() == -1) {
        perror("sem_initialize");
        exit(1);
      }
      if(s2 = sem_initialize() == -1) {
        perror("sem_initialize");
        exit(1);
      }

      // acquire semaphores : les prochains sem_acquire seront bloques
      if(p1 = sem_acquire(s1) == -1) {
        perror("sem_acquire");
        exit(1);
      }
      if(p2 = sem_acquire(s2) == -1) {
        perror("sem_acquire");
        exit(1);
      }

      // fork pour creer fils 1 et fils 2
      if(pid1 = fork() == -1) {
        perror("fork");
        exit(1);
      } else {
        // fils 1
        sem_acquire(s1);
        write(1,"fils 1",6);
        sem_release(s1);
        exit(0);
      }
      if(pid2 = fork() == -1) {
        perror("fork");
        exit(1);
      } else {
        // fils 2
        sem_acquire(s2);
        write(1,"fils 2",6);
        sem_release(s2);
        exit(0);
      }

      // pere
      if(pid1 >= 1) {
        sem_release(s1);
        wait(&status); // on attend la fin du fils 1
        sem_release(s2); // on reveille le fils 2
        wait(&status);
      }
      else if(pid2 >= 1){
        sem_release(s2);
        wait(&status); // on attend la fin du fils 2
        sem_release(s1); // on reveille le fils 1
        wait(&status);
      }
      write(1,"\n",1);
      return 0;
    }
    \end{lstlisting}

\newpage
\section{Implémentation}
    \subsection{Structure du sémaphore}
      Pour implémenter la structure du sémaphore, plusieurs modifications sont à faire dans les fichiers sched.h et fork.c.\\

      La structure \textit{task\_struct}, qui se trouve dans le fichier sched.h, doit implémenter de nouveaux attributs :
      \begin{itemize}
        \item un \textbf{tableau d'entiers} représentant l'ensemble des identifiants des sémaphores.
        \item un \textbf{tableau de sémaphores}.\\
      \end{itemize}

      Le sémaphore aura une structure définie comme suit :
      \begin{itemize}
        \item un \textbf{entier} représentant l'identifiant du sémaphore
        \item un \textbf{nombre} représentant le nombre de ressources disponibles pour ce sémaphore
        \item une \textbf{file d'attente} de processus.\\
      \end{itemize}
      Pour déterminer un numéro d'identification pour le sémaphore, on initialise une variable globale à 0 qui s'incrémente à chaque appel de l'appel système \textit{ss\_initialize}.\\

      De plus, les fonctions \textit{do\_clone}() et \textit{exit}() doivent être modifiées :
      \begin{itemize}
        \item dans \textit{do\_clone}(), les processus fils doivent hériter des sémaphores du processus père.
        \item dans \textit{exit}(), il faut vérifier s'il y a des sémaphores qui ne sont plus utilisés et les supprimer.\\
      \end{itemize}

    \subsection{Appels systèmes}
      \subsubsection{initialize}
        L'appel système se fait de la manière suivante :
        \begin{itemize}
          \item Enregistre dans une variable \textit{result} la variable globale représentant le nombre de sémaphore créé à cet instant.
          \item Incrémente la variable globale.
          \item Enregistrer l'identifiant dans le processsus.
          \item Renvoie result.
        \end{itemize}
      \subsubsection{acquire}
        L'appel système se fait de la manière suivante :
          \begin{itemize}
            \item Recherche le sémaphore dans task\_struct.
            \item Décrémente la valeur du sémaphore.
            \item Dans le cas où cette valeur est strictement inférieur à 0, on bloque le processus et on le met dans une file d'attente.
            \item Renvoie 0 s'il n'y a eu aucune erreur, sinon -1.
          \end{itemize}
      \subsubsection{release}
        L'appel système se fait de la manière suivante :
          \begin{itemize}
            \item Recherche le sémaphore dans task\_struct.
            \item Incrémente la valeur du sémaphore.
            \item Dans le cas où la file d'attente est non vide, on débloque le processus en tête de file.
            \item Renvoie 0 s'il n'y a eu aucune erreur, sinon -1.
          \end{itemize}
      \subsubsection{destroy}
        L'appel système se fait de la manière suivante :
          \begin{itemize}
            \item Recherche le sémaphore dans task\_struct.
            \item Désalloue la file d'attente liée au sémaphore.
            \item Détruit le sémaphore.
            \item Renvoie 0 s'il n'y a eu aucune erreur, sinon -1.
          \end{itemize}
      \subsubsection{debug}
        L'appel système se fait de la manière suivante :
          \begin{itemize}
            \item Recherche le sémaphore dans task\_struct.
            \item Affichage d'informations concernant le sémaphore, comme les pids des processus dans la file d'attente.
            \item Renvoie le nombre de processus dans la file d'attente s'il n'y a eu aucune erreur, sinon -1.
          \end{itemize}
\newpage
\section{Recette}
L'implémentation d'un sémaphore local sera testée par des programmes C. Nous allons écrire deux programmes afin de tester le sémaphore de différentes manières.
\subsection{Premier programme de test}
  \paragraph{Objectif\\}
  Ce premier programme permet de tester la création et la destruction d'un sémaphore.

  \paragraph{Implémentation\\}
  Le programme sera de la forme suivante :
  \begin{lstlisting}[language=C]
  int main(int argc, char* argv) {
    int sem;

    // init semaphores
    if(sem = sem_initialize() == -1) {
      perror("sem_initialize");
      exit(1);
    }

    // debug -> waitlist empty
    if(nbWaiting = sem_debug(sem) == -1) {
      perror("sem_debug");
      exit(1);
    }

    // acquire semaphore
    sem_acquire(sem);

    // debug -> resource blocked
    if(nbWaiting = sem_debug(sem) == -1) {
      perror("sem_debug");
      exit(1);
    }

    // release semaphore
    sem_release(sem);

    // debug -> resource released
    if(nbWaiting = sem_debug(sem) == -1) {
      perror("sem_debug");
      exit(1);
    }

    // destruct semaphore
    sem_destroy(sem);

    // debug -> error because semaphore sem not found (has been destroyed)
    if(nbWaiting = sem_debug(sem) == -1) {
      perror("sem_debug");
      exit(1);
    }

    return 0;
  }
  \end{lstlisting}
  \paragraph{Résultat attendu\\}
  Grâce à la fonction \textit{sem\_debug} utilisée dans le programme de test on doit obtenir l'affichage décrivant les cas suivants sur la console :
  \begin{itemize}
    \item un semaphore créé, avec une ressource de disponible et aucun processus en file d'attente
    \item le semaphore doit être bloqué, le nombre de ressources disponibles doit être nul
    \item le semaphore a été libéré, le nombre de ressources disponibles doit être de 1
    \item le semaphore a été détruit, une erreur doit s'afficher car on essaie de lancer la fonction \textit{sem\_debug} avec l'id d'un semaphore qui n'existe plus
  \end{itemize}

  \newpage
  \subsection{Deuxième programme de test}
  \paragraph{Objectif\\}
  Ce programme permet de tester l'implémentation des semaphores lors de la création d'un processus fils.
  \paragraph{Implémentation\\}
  Le programme sera de la forme :
  \begin{lstlisting}[language=C]
  int main(int argc, char* argv) {
    int sem, status, pid1;

    // init semaphore
    if(sem = sem_initialize() == -1) {
      perror("sem_initialize");
      exit(1);
    }

    // fork
    if(pid1 = fork() == -1) {
      perror("fork");
      exit(1);
    }
    if(pid1 == 0){
      // fils 1
      sem_acquire(sem);
      write(1,"child acquired sem\n",19)
      sleep(2); // waiting for parent to acquire too
      sem_debug(sem); // parent is in the waitlist
      sem_release(sem);
      sem_debug(sem); // empty waitlist, no free resource
      exit(0);
    }
    else {
      // pere
      sleep(3); // waiting to be sure child will acquire first
      sem_acquire(sem);
      write(1,"parent acquired sem\n",20)
      wait(&status); // waiting for child
      sem_release(sem);
      sem_debug(sem); // empty waitlist, 1 free resource
    }
    return 0;
  }
  \end{lstlisting}
  \paragraph{Résultat attendu\\}
  Grâce à la fonction \textit{sem\_debug} utilisée dans le programme de test on doit obtenir l'affichage décrivant les cas suivants sur la console :
  \begin{itemize}
    \item le père est dans la liste d'attente du semaphore, car le fils fait un \textit{sem\_acquire} en premier
    \item le fils a libéré le semaphore, le père a pu faire \textit{sem\_acquire} donc la file d'attente est vide et il n'y a aucune ressource disponible car elle est utilisée par le père
    \item le père a libéré le semaphore, la file d'attente est vide et le nombre de ressources disponibles est de 1
  \end{itemize}

  \newpage
  \subsection{Troisième programme de test}
  \paragraph{Objectif\\}
  Ce programme permet de tester l'ensemble des fonctionnalité d'un semaphore local hérité par les fils d'un processus.
  \paragraph{Implémentation\\}
  Un processus père va créer 2 fils, et l'un de ces deux fils va lui-même créer un fils. Ces 4 processus modifieront la valeur d'une variable critique. Deux processus vont incrémenter cette valeur 1000 fois, les deux autres vont la decrémenter 1000 fois.
  Le programme sera de cette forme :
  \begin{lstlisting}[language=C]
  int main(int argc, char* argv) {
    int sem, status, pid1, pid2, pid3, i;

    // init and print critical value
    int criticalValue = 42;
    printf("Initial value is %d\n", criticalValue);

    // init semaphore
    if(sem = sem_initialize() == -1) {
      perror("sem_initialize");
      exit(1);
    }

    // fork
    if(pid1 = fork() == -1) {
      perror("fork");
      exit(1);
    }
    if(pid1 == 0){
      // fils 1
      sem_acquire(sem);
      write(1,"fils 1 increments value\n",24);
      for(i = 0; i < 1000; i++)
        criticalValue += 1;
      sem_release(sem);
      exit(0);
    }

    if(pid2 = fork() == -1) {
      perror("fork");
      exit(1);
    }
    if(pid2 == 0) {
      // fils 2
      if(pid3 = fork() == -1) {
        perror("fork");
        exit(1);
      }
      if(pid3 == 0) {
        // fils 3
        sem_acquire(sem);
        write(1,"fils 3 decrements value\n",24);
        for(i = 0; i < 1000; i++)
          criticalValue -= 1;
        sem_release(sem);
        exit(0);
      }
      else {
        // fils 2
        sem_acquire(sem);
        write(1,"fils 2 increments value\n",24);
        for(i = 0; i < 1000; i++)
          criticalValue += 1;
        sem_release(sem);
        exit(0);
      }
    }

    // pere
    sem_acquire(sem);
    write(1,"pere decrements value\n",22);
    for(i = 0; i < 1000; i++)
      criticalValue -= 1;
    sem_release(sem);

    // waiting for children
    wait(&status);
    wait(&status);

    // print final critical value
    printf("Final value is %d\n", criticalValue);
    return 0;
  }
  \end{lstlisting}
  \paragraph{Résultat attendu\\}
  Le résultat attendu lors du lancement de ce programme C est un affichage sur la console de la valeur de la variable critique en début de programme ainsi que cette valeur à la fin du programme : la valeur doit être inchangée.

\end{document}
