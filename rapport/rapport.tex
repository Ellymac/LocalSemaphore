\documentclass[12pt]{article}

\usepackage[sfdefault]{ClearSans}
\usepackage[utf8]{inputenc}
\usepackage[T1]{fontenc}
\usepackage[francais]{babel}
\usepackage{color}
\usepackage[top=2cm, bottom=2cm, left=2cm, right=2cm]{geometry}
\usepackage{eurosym}
\usepackage{graphicx}
\usepackage{fancybox}

\pagestyle{plain}
\title{Sémaphore local}
\author{Anthony Araye et Camille Schnell}
\date{21 février 2018}
\begin{document}
\maketitle
\renewcommand{\contentsname}{Sommaire}
\tableofcontents
\newpage
\section{Introduction}
    \textit{Cinq philosophes se trouvent autour d'une table avec en face d'eux un plat de spaghettis et à gauche de chaque plat se trouve un couvert (une fourchette ou un couteau).
    Un philosophe ne possède que trois états :}
    \begin{itemize}
        \item \textit{penser pendant un temps indéterminé}
        \item \textit{être affamé pendant un temps déterminé et fini}
        \item \textit{manger pendant un temps déterminé et fini.}
    \end{itemize}

    \textit{Quand un philosophe a faim, il se met en état "affamé" et va attendre que les couverts situés de chaque côté de son assiette soient libres pour pouvoir manger.
    Dans le cas ou l'un des deux couverts n'est pas libre, le philosophe se met en état de famine pendant un temps déterminé avant de revérifier leur disponibilité.}

    Cette situation représente en réalité le problème du "dîner des philosophes" énoncé par Dijkstra. \\ \\

    En informatique, lorsque deux programmes exécutés en parallèle souhaitent accéder à la même donnée au même moment, cela peut créer des conflits, notamment si l'un des deux modifie la donnée. \\

    Le problème des lecteurs et des rédacteurs énoncé par Dijkstra illustre bien la situation :

    \textit{Soit une base de données pouvant être accédée par des lecteurs qui vont simplement consulter les données, et des rédacteurs qui vont les modifier.
    Les contraintes sont les suivantes :}
    \begin{itemize}
      \item \textit{Plusieurs lecteurs peuvent accéder en même temps à la base de données.}
      \item \textit{Lorsqu'un rédacteur modifie la base de donnée, personne d'autre ne peut y accéder (ni les lecteurs, ni un autre rédacteur).} \\
    \end{itemize}

    L'implémentation d'un système de sémaphores afin de gérer des sections critiques va permettre de répondre au problème ci-dessus. En effet, le sémaphore va agir comme un verrou et mettre en attente des programmes ou threads s'ils souhaitent accéder à une section critique en cours d'utilisation par un ou plusieurs autres programmes ou threads.


\newpage
\section{Manuel d'utilisation}
    \subsection{Tutoriel d'utilisation}
    \newpage
    \subsection{initialize}
        \paragraph{Nom\\}
        sem\_initialize - Initialise un sémaphore
        \paragraph{Synopsis\\}
        \textbf{\#include <sem.h>}

        \textbf{int sem\_initialize(sem *}\textit{sem}\textbf{, int }\textit{value}\textbf{)}
        \paragraph{Description\\}
        \textbf{sem\_initialize}() alloue et initialise \textit{sem} avec \textit{value} comme valeur initiale. \textit{value} doit être supérieur ou égale à 0.

        \paragraph{Valeur renvoyée\\}
        \textbf{sem\_initialize}() renvoie 0 dans le cas où le sémaphore a bien été créé, -1 si une erreur a été levé.
        \paragraph{Erreurs\\}
        EFAULT : \textit{value} est strictement inférieur à 0.
    \newpage
    \subsection{acquire}
        \paragraph{Nom\\}
        sem\_acquire - Décrémente la valeur du sémaphore
        \paragraph{Synopsis\\}
        \textbf{\#include <sem.h>}

        \textbf{int sem\_acquire(sem *}\textit{sem}\textbf{)}
        \paragraph{Description\\}
        \textbf{sem\_acquire}() décrémente la valeur du sémaphore \textit{sem}. Dans le cas où cette valeur est inférieur ou égale à 0, alors il bloque le processus courant et le met dans la file d'attente associée au sémaphore.

        \paragraph{Valeur renvoyée\\}
        \textbf{sem\_acquire}() renvoie 0 dans le cas où le processus s'est bien effectué, -1 si une erreur a été levé.
        \paragraph{Erreurs\\}
        ?

    \newpage
    \subsection{release}
        \paragraph{Nom\\}
        sem\_release - Incrémente la valeur du sémaphore
        \paragraph{Synopsis\\}
        \textbf{\#include <sem.h>}

        \textbf{int sem\_release(sem *}\textit{sem}\textbf{)}
        \paragraph{Description\\}
        \textbf{sem\_release}() incrémente la valeur du sémaphore \textit{sem}. Dans le cas où la file associée à \textit{sem} est non vide, alors il réveille le processus en tête de file et saute la tête de la file.

        \paragraph{Valeur renvoyée\\}
        \textbf{sem\_release}() renvoie 0 dans le cas où le processus s'est bien effectué, -1 si une erreur a été levé.
        \paragraph{Erreurs\\}
        ?
    \newpage
    \subsection{destroy}
        \paragraph{Nom\\}
        sem\_destroy - Détruit le sémaphore
        \paragraph{Synopsis\\}
        \textbf{\#include <sem.h>}

        \textbf{int sem\_destroy(sem *}\textit{sem}\textbf{)}
        \paragraph{Description\\}
        \textbf{sem\_destroy}() détruit le sémaphore \textit{sem} et désalloue la file d'attente.

        \paragraph{Valeur renvoyée\\}
        \textbf{sem\_destroy}() renvoie 0 dans le cas où le sémaphore a bien été détruit, -1 si une erreur a été levé.
        \paragraph{Erreurs\\}
        ?
    \newpage

\section{Implémentation}
\newpage
\section{Recette}
L'implémentation d'un sémaphore local sera testée par des programmes C. Nous allons écrire trois programmes afin de tester le sémaphore de différentes manières.
\subsection{Premier programme de test}
  \paragraph{Objectif\\}
  Dans ce premier programme, les sémaphores vont agir comme des mutex (initialisés à 1). L'objectif est à partir de deux threads de naviguer entre deux sections critiques l'une après l'autre.
  \paragraph{Implémentation\\}
  Deux threads \textit{tPing} et \textit{tPong} seront lancés. Il y aura deux sections critiques contenant chacune un affichage à l'écran indiquant la section critique dans laquelle le thread se situe ("ping" ou "pong"). \\
  Le sémaphore \textit{pong} sera initialisé à 0 pour commencer par l'affichage de "ping". Le sémaphore \textit{ping} sera ainsi initialisé à 1.
  \paragraph{Résultat attendu\\}
  Le résultat attendu lors du lancement de ce programme C est un affichage sur la console d'une suite de "ping pong ping pong...".
\newpage
\subsection{Deuxième programme de test}
  \paragraph{Objectif\\}
  Dans le deuxième programme de test, les sémaphores seront utilisés pour permettre de résoudre le problème des lecteurs et rédacteurs énoncé en introduction.
  \paragraph{Implémentation\\}
  \paragraph{Résultat attendu\\}
  Le résultat attendu lors du lancement de ce programme C est :
  \begin{itemize}
    \item Lorsqu'un lecteur accède à la base de données, on l'affiche sur la console avec le nombre total actuel de lecteurs. De même lorsqu'un lecteur quitte la base de données.
    \item Lorsqu'un rédacteur accède à la base de données, on l'affiche sur la console. De même lorsque le rédacteur quitte la base de données.
  \end{itemize}
\newpage
\subsection{Troisième programme de test}
  \paragraph{Objectif\\}
  \paragraph{Implémentation\\}
  \paragraph{Résultat attendu\\}





\end{document}
